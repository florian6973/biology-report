\documentclass{article}

\input{includes}

\usepackage{biblatex}
\addbibresource{biblio/biblio.bib}

%\usepackage[backend=bibtex]{biblatex}
\usepackage[nottoc, notlof, notlot]{tocbibind}

\begin{document}

    \title{\Large ES2A EEP-02  \\[0.5cm]
        \bf\Large Vaccine strategies}
\author{\large Florent Pollet \ \\}
\date{\large\today}

\makeatletter
    \begin{titlepage}
        \begin{center}
	   { \includegraphics[width=12cm]{imgs/mp_logo.png}}
	   {\ \\ \ \\}
        \vbox{}\vspace{2cm}
            {\@title }\\[1cm] 
            %{ \includegraphics[width=7cm]{imgs/cover.PNG}}\\[1cm]
            {\@author}

            {\large \ \\ Supervisor: \bf Véronique Stoven\\ \ \\}
            {\@date\\}

        \end{center}

    \end{titlepage}

        
    \begin{remerciements}
        
    \end{remerciements}


    \tableofcontents


    \newpage

    \makeatother

    \section{Introduction}

    Vaccines represent an amazing invention which saved millions of lives in the world since 2000 \autocite{HowManyLives2021}.
    Even if vaccination is not new, the recent pandemic of COVID-19 brought to the forefront this field, generating great advances but also important debates about safety, ethics and politics.
    It is necessary to take into account all these changes and to adapt to be avoid a backward step and the rejection of these opportunities. 
    
    Therefore it would be interesting to better understand the recent evolutions in vaccine strategies and technologies and the consequences it may have on society.
    
    Firstly, I will briefly recall the history of vaccination, detailing the different types of vaccines which have been developed.
    Secondly, I will present the technology of mRNA (messenger ribonucleic acid) vaccines which are currently the main source of progress.
    Thirdly, I will observe how society is integrating these developments and what issues remain to be solved.

    This theme is closely linked to the mechanics of the immune system, in medicine, which will not be detailed in this article due to the needed knowledge in medicine.
    In any case, I would like to thank Ms. Stoven for her molecular biology course and her help in the writing of this article.

    %Blabla. Belle oeuvre : \autocite{zhangMulticladeEnvGag2021}, blabla.

    \section{A brief history of vaccines}

        \subsection{Discovery of the principle}

            The principle of vaccination could come from the idea of mithridatism, that is to say the ability to gain protection against a poison by taking several benign amounts.
            
            Variolization, which corresponds to injecting smallpox pustules to gain immunity against smallpox, began in China around the 10\up{th} century \autocite{canouiHistoryPrinciplesVaccination2019}. 
            Before the scientist Jenner, several persons realized variolization in England. However, in 1798, this is Jenner who made a link between cowpox and smallpox: 
                cowpox could an attenuated version of smallpox.            

            The idea of vaccination is to create an individual, long-term and efficient protection against a pathogen (like a virus or a bacteria), without causing serious symptoms.
            This is made possible thanks to the memory of human's immune system.

            The word "vaccination" comes from the Latin word "vacca", which means "cow".


            \subsubsection{A quick look on the immune system}

                %humoral and cellular response

                %T lymphocyte

                %B lymphocyte : humoral

                % based on the memory, igM (non-specific antibody), quicker answer and more important, igG, more specific
                % cellular response too but more complicated to see

                % antibodies production: 6 weeks after, membory B cell, increase till 15 weeks

                % different objectives if incubation period is short or not (preumocoque vs HBV)

            % variole eradication

            % lymphocyte T & B mémoire

            % schéma rappel et courbe antibodies dans les stratégies de vaccin

        \subsection{First generation of vaccines}
            
            It is only between 1870 and 1885 with Pasteur's works (helped by Emile Roux and Emile Duclaux, and based on Robert Koch's findings), that the first vaccines were developed \autocite{plotkinHistoryVaccination2014}.
            The first official vaccines that he conceived were against chicken cholera, anthrax, swine erysipelas and then rabies.

            The first generation of vaccines consists in pathogens which are live, weakened or killed. Live weakened vaccines are the most immunogenic,
                but they should be prepared carefully. Despite being efficient, there are still high risks using them if virus are not attenuated enough:
                one can develop the disease and transmit it to an immunocompromised person. Otherwise, vaccines can be killed for example with heat/chemical treatment, and they
                are less dangerous, however there is a need for adjuvants (from the Latin word "adjurvare", which means "to help") 
                to make them immunogenic, and the response is mainly humoral and less cellular.

            Adjuvants can be very diverse and numerous as well as their mechanism of action: killed mycobacteria, oils, 
                aluminium salts, microparticles, squalanes, ligands of 
                PRRs... They aim at amplifying the reaction for a whole population and each individual (especially old people whose immune system is less dynamic) and 
                at reducing the quantity of active substance needed (dose sparing). The dosage should be very cautious not to hide the active substance.

            % how to to prepare these vaccines ?

            To prepare a live weakeaned vaccine, the virus or the bacteria is replicated in adverse conditions several times.



            % explain principle
            % some schema cool
            % majority : humoral response which can be measured to know if something is protected against a disease

            To prepare those vaccines, cell culture is often needed before weakening the virus with different techniques such as 
            
            %https://www.pnas.org/content/111/34/12283

            Thanks to the development of immunology and microbiology, in particular the ability to isolate pathogens, vaccines were better understood
                and it led to the development of a new type of vaccines.

            The method of production of these vaccines also evolved over time. With genetic engineering and reverse genetics, that is to say,  
                contrary to forward genetic, the ability to know which phenotypes can be controlled by different genetic sequences,
                it allows a better treatment of viruses to select inactivated vaccines.


        \subsection{Second generation of vaccines}

            This generation demands more development and technological advances,
                to use subunits of viruses such as protein antigens or recombinant protein components (antigens for instance).
            There is also the need for adjuvant, which are substances which trigger a powerful immune response ; otherwise the vaccine would not be effective. 
            The response is again mainly humoral: sometimes it cannot activate Toll-Like Receptors on dentritic cells needed for a complete response.


            An anatoxin is a toxin which has lost his toxic power thanks to heat and formaldehyde, but not totally its immunogenic ability.

        \subsection{Towards a third generation}

            %genetic vaccines

        % not only prophylathic but therapeutic

    \section{mRNA vaccines: a promising technology}

        \subsection{Principle}

        \subsection{Development}

        % eradication of smallpox and polyomyelite

        % ideal vaccine before ARNm possibilities:  alive weakeaned vaccine, injected into the mucous membranes, to stimulate IgA production

        % usual : injection sous-cutanée, intradermique (peau) ou intramusculaire (plus profond), intraveineuse
        %https://www.google.com/url?sa=i&url=https%3A%2F%2Fwww.infirmiers.com%2Fetudiants-en-ifsi%2Fcours%2Fadministration-d-un-vaccin.html&psig=AOvVaw1kw-yPymOdrfp6v5eTiI-n&ust=1643840900387000&source=images&cd=vfe&ved=0CAsQjRxqFwoTCLDcp9nG3_UCFQAAAAAdAAAAABAD
        % empirical choice before understanding that the aim is to choose the best dentritic cells


        \subsection{Advantages and drawbacks}

    \section{New stakes for vaccines in the 21\up{st} century}

        \subsection{World inequalities}

        \subsection{Public policies}

            Firstly it is an individual protection.

            Secondly it protects the whole population thanks to herd immunity: the pathogen is less likely to be transmitted ; otherwise it would be exponential,
                as people observed for the Covid-19 pandemic.

        %t. Cinq vaccins sont recommandés pour 
%les personnes de plus de 60 ans : antigrippal, antitétanique, 
%antidiphtérique, anticoquelucheux et antipneumococcique. Par rapport aux s

%augmenter les doses d'antigènes 
%pour augmenter la présentation par les cellules dendritiques, utiliser de nouveaux adjuvants pour recruter plus 
%de cellules immunocompétentes (tel que des dérivés de 
%saponine, des liposomes ou des ligands des TLR), varier 
%les voies d'immunisation en privilégiant la voie muqueuse 
%(intra-nasale ou intra-dermique)

        \subsection{New uses and technical challenges}

        % prophylatic vaccines VS therapeutic vaccines (avoid expression of the disease)
        % active substance is immunogenic

    \section{Conclusion}

    % science is not enough
    % rabelais
    % vaccine acceptance is as important as vaccine development

    % what is a vaccine ?

    % a short history : first generation (whole-organism vaccines), second generation (reduce risks from live vaccines = subunit vaccines/toxoid/recombinant),
                        % third generation (RNA/DNA vaccine)
                        %In 2021, Katalin Karikó and Drew Weissman received Columbia University's Horwitz Prize for their pioneering research in mRNA vaccine technology
                        %operatioin warm speed

    % monovalent & polyvalent (more complex : interferences, strength of the immune response)
    % only student monovalent
    % only nanogram or microgram immogen
    % adjuvant (alum) to boost or to store (preservatives, thiomersal/phenoxyethanol, formaldehyde in influenza) -> risk of allergy
    %% excipients also : antibiotics, egg protein, kill 
    % names : abbreviation

    % legislation
    % patent : stakes : licensing
    % vaccine acceptance by population (save live, economic)
    % manufacturing, distribution, losgitical factor : prioritize
    % state role and WHO role
    % record for adverse eeffects...
    % EMA/FDA



    % which phase for a vaccine development
    %scientific and then safety 
    % use and scheduling ? many vaccine when young, need combination of injections
    % different  phases
    % paradox of economic development of vaccines : role of government, universities and non-profit organizations
    % need infrastructure and workforce
    % The large number of vaccines and boosters recommended (up to 24 injections by age two) has led to problems with achieving full compliance.
    % to healthy people : huge quality needed
    % build : bioreactor
    % vials = flacons
    % filling vials : difficult step

    % key players & firms
    % 10/15 years official development
    % from injection to oral vaccines (ouverture)

    % link to veterinary medicine

    % revoir test ELISA principe
    % next stakes : vaccine resistance (same as antimicrobial resistance) (immune evasion) -> covid, this is the case, if total protection less likely than only against serious forms
    % transgenic plant
    

    %https://biontech.de/covid-19-portal/mrna-vaccines (interesting schema)
    %https://sitn.hms.harvard.edu/flash/2015/rna-vaccines-a-novel-technology-to-prevent-and-treat-disease/
    % Bill & Melinda Gates Foundation invested $53 million in the German company, CureVac, which specializes in the development of these vaccines
    % perspective for the future as regard influenza vaccines : influenza vaccine, whereas the company CureVac claims that RNA-based vaccines could be manufactured in less than two months at a lower production cost, (harvard) + cancer vaccines
    %

    \newpage

    \appendix

    \section {Source code}
    The source code of this document is available at \url{https://github.com/florian6973/biology-report}.
    
    \nocite{*}
    \printbibliography
    
\end{document}